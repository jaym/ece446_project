\documentclass[12pt, a4paper]{article}

\usepackage[top=1in, bottom=1in, left=1in, right=1in]{geometry}
%\usepackage{setspace}
%\onehalfspacing
\usepackage{graphicx}
\usepackage{float}

\usepackage{subfig}
\usepackage{amsmath}
\usepackage{url}
\usepackage{listings}
\usepackage{color}
\usepackage{multirow}
\usepackage{algorithmic}
\usepackage{algorithm}
\usepackage{wrapfig}


\newcommand{\HRule}{\rule{\linewidth}{0.5mm}}

\begin{document}

\begin{titlepage}

  \begin{center}
  \textsc{\Large ECE 446-01 Final Project}\\[0.5cm]


  % Title
  \HRule \\[0.4cm]
  { \huge \bfseries Asynchronous Serial Communication}\\[0.4cm]
  \small \url{http://github.com/whois/ece446_project}
  \HRule \\[1.5cm]

\end{center}

  \begin{minipage}{0.4\textwidth}
    \begin{flushleft} \large
      \emph{Author:}\\
      Jay \textsc{Mundrawala}
    \end{flushleft}
  \end{minipage} \\[0.4cm]


  \vfill

  \begin{minipage}{0.9\textwidth}
    \begin{flushright} \large
      \emph{Due Date:} \\
      December 9, 2009      
    \end{flushright}
  \end{minipage}

%\end{center}

\end{titlepage}
\tableofcontents
\newpage

\begin{abstract}
  This paper describes the implementation of a simple data communication system using the
  RS-232 protocol and the Xilinx Spartan 3E experimental board.
\end{abstract}
\section{Introduction}
The objective of this project is to become familiar with data communication concepts. This project
involves the implementation of an asynchronous serial data communication protocol, RS-232. This is
done using the Xilinx Spartan 3E experimental board.

Figure \ref{fig:sys_blk} shows the entire system. It consists of a transmitter, receiver, clock generator,
RS-232 voltage converter, LCD module, and a set of switches. The transmitter is responsible for
taking one byte of data and transmitting over a serial line. It is connected to a voltage converter.
The RS-232 protocol specifies that a logic H is transmitted as -5V, while a logic L is transmitter as
5V. The converter is responsible for taking care of this, and is provided on the Spartann 3E experimental
board.\cite{xilinxug} The receiver takes data from the serial line and stores it. Once a byte has been received, it
is available to be retrieved. The clock generator is used to generate the appropriate clocks. For
the transmitter, it generates a clock that is 16 times slower than that of the receiver. The clock
generator has a control input which allows the selection of transmitting at 1bps or 10bps. The LCD
module displays the data on the transmit lines, and the data that has been received by the receiver.
The switches are used to control the circuit.

\begin{figure}
  \caption{System Block Diagram}
  \centering
  \includegraphics[width=0.8\textwidth]{images/sys_block.eps}
  \label{fig:sys_blk}
\end{figure}


\section{Asynchronous Transmitter/Receiver}
\subsection{Asynchronous Transmitter}
The transmitter, as described above, takes one byte of data, and shifts it onto the serial data
transmission line at its given clock rate. For this project, it can have either a 1Hz clock running
it, or a 10Hz clock.
\subsubsection{Block Diagram}
Figure \ref{fig:tran_blk} shows the block diagram for the transmitter. The transmitter has three inputs
and two outputs. Data is a one byte value that contains the data to be transmitted. Start tells the
transmitter to begin transmission. CLK is used to time the transmission of data. Tx\_D is the serial line
onto which the data is shifted. RDY specifies that the transmitter is not sending anything and
is ready to send another byte of data.

\subsubsection{Algorithm and Flowchart}
\begin{figure}
  \caption{Transmitter Block Diagram}
  \centering
  \includegraphics{images/transmitter_block}
  \label{fig:tran_blk}
\end{figure}
\begin{figure}
  \centering
  \includegraphics[width=0.25\textwidth]{images/transmitter_fc.eps}
  \caption{Transmitter Flowchart}
  \label{fig:tran_fc}
\end{figure}
Figure \ref{fig:tran_fc} shows the flowchart for the transmitter block. The algorithm is very simple. 
The idea here is that when the user requests that data be transmitted,
it first gets latched internally. This way, external changes can be ignored. The first thing the 
transmitter must send is a start bit, which is 0. The transmitter outputs a 0 for RDY and until
the data has been transmitted. The internal counter is reset at this point. Next, 1 bit is transmitted
at each of the next 8 clock cycles. Next, a stop bit is transmitted, its value being 1. After this,
the transmitter starts over and awaits the next start signal from the user. The following is the 
pseudocode:
%\begin{algorithm}
  %\caption{Transmitter algorithm}
  %\label{alg:tran}
\begin{algorithmic}
  \LOOP
    \STATE $TX\_D \gets 1$
    \STATE $RDY \gets 1$
    \IF {$start = 1$}
    \IF{ $CLK=1$}
        \STATE $TX\_D \gets 0$ \COMMENT {Send start bit}
        \STATE $RDY \gets 0$
        \STATE $bitcount:=0$
        \STATE $data \gets input$ \COMMENT {Latch Data}
        \WHILE{$bitcount \neq 8$}
            \IF {$CLK = 1$}
                \STATE $TX\_D \gets data(0)$ \COMMENT {Transfer LSB}
                \STATE $RDY \gets 0$
                \STATE $bitcount = bitcount + 1$
                \STATE $data \gets 1$ \& data(7 downto 1) \COMMENT {Shift}
              \ENDIF
        \ENDWHILE
        \IF{$CLK = 1$}
            \STATE $TX\_D \gets 1$ \COMMENT {Send stop bit}
            \STATE $RDY \gets 0$
          \ENDIF
      \ENDIF
    \ENDIF
  \ENDLOOP
\end{algorithmic}
%\end{algorithm}

\subsubsection{VHDL Implementation}
\lstset{ %
language=VHDL,                % choose the language of the code
basicstyle=\footnotesize,
numbers=left,                   % where to put the line-numbers
stepnumber=1,                   % the step between two line-numbers. If it's 1 each line will be numbered
numbersep=10pt,                  % how far the line-numbers are from the code
showspaces=false,               % show spaces adding particular underscores
showstringspaces=false,         % underline spaces within strings
showtabs=false,                 % show tabs within strings adding particular underscores
frame=none,	                % adds a frame around the code
tabsize=2,	                % sets default tabsize to 2 spaces
captionpos=b,                   % sets the caption-position to bottom
breaklines=true,                % sets automatic line breaking
breakatwhitespace=false        % sets if automatic breaks should only happen at whitespace
}
\lstset{caption=Transmitter,label=lst:tran}
\lstinputlisting{src/transmitter.vhdl}

\subsubsection{Simulation}
Figure \ref{fig:transsim} shows the simulation results. Listing \ref{lst:trantb} shows the testbench
used to run this simulation.
\begin{figure}[H]
  \centering
  \includegraphics[width=1.0\textwidth]{images/transsim.eps}
  \caption{Transmitter Simulation Waveform}
  \label{fig:transsim}
\end{figure}


\subsection{Asynchronous Receiver}
The receiver reads data off of the serial line one bit at a time. These bits are shifted into a register,
and once eight data bits have been read, the data becomes ready. The receiver will output a frame error in
the case that the final bit is not the value of the designated stop bit.
\subsubsection{Block Diagram}
Figure \ref{fig:rec_blk} shows the block diagram for the receiver. It has two inputs. First, RX\_D. RX\_D is
the line used for serial communication. Next, there is CLK. CLK is a clock that pulses 16 times faster than
the transmitters clock. The data is then sampled after 8 ticks. The reason for this is so that the data on
the serial line is stable by the time that it is sampled. The DATA line is used to output the byte read on
RX\_D. RDY indicates that the receiver is ready to receive more data. FERR indicates that there has been a
framing error while receiving the data.
\begin{figure}[H]
  \caption{Receiver Block Diagram}
  \centering
  \includegraphics{images/receiver_bock}
  \label{fig:rec_blk}
\end{figure}
\subsubsection{Algorithm and Flowchart}
Figure \ref{fig:rec_fc} shows the flowchart for the asynchronous receiver. The algorithm is not much more
complicated than that of the transmitter. The FSM begins by outputing $RDY=1$ and $FERR=0$ to indicate that
the receiver is ready to receive and no error is present. It then waits for a falling edge on RX\_D. This
indicates that a start bit was sent to the receiver. After the falling edge is detected, the FSM waits for
8 clock cycles, and then samples RX\_D. This gives enough time for RX\_D to stabalize. If RX\_D is still low,
it means that this is very likely a legitimate data transfer. Otherwise, the receiver will go back to searching
for a falling edge on RD\_D. If RX\_D was low, the receiver will initialize two counters. One ticks at every 
clock cycle. This clock is used to determine when it is appropriate to sample that data, and in this case
is every 16 ticks. The second counter ticks once after each bit has been sampled. This is used to keep
track of when it is time to search for the stop bit. If the stop bit is not 1, FERR goes high for 16 clock cycles.
This is not shown in the FSM, however a error state was added that lasts 16 clock cycles in order to see
the frame error. If RX\_D did equal one, the receiver can go back to waiting for more data. The following is
the psuedocode for the receiver:
\begin{algorithmic}
  \LOOP
    \STATE $RDY \gets 1$
    \STATE $FERR \gets 0$
    \IF{ $CLK = 1$ } 
    \IF{ $RX\_D = 0$}
        \STATE WAIT 8 CLK CYCLES
        \IF{ $RX\_D = 0$}
            \STATE $RDY \gets 0$
            \STATE $FERR \gets 0$
            \STATE $bitcount:=0$
            \WHILE{ $bitcount < 8$}
                \STATE WAIT 16 CLK CYCLES
                \STATE $data \gets $ data(7 downto 1) \& RX\_D
                \STATE $bitcount = 0$
            \ENDWHILE
            \IF{$RX\_D = 0$}
                \STATE $FERR \gets 1$
                \STATE WAIT 16 CLK CYCLES
            \ENDIF
        \ENDIF
    \ENDIF
    \ENDIF
  \ENDLOOP
\end{algorithmic}

\begin{figure}[H]
  \caption{Receiver Block Diagram}
  \centering
  \includegraphics{images/receiver_fc.eps}
  \label{fig:rec_fc}
\end{figure}
\subsubsection{VHDL Implementation}
\lstset{caption=Receiver,label=lst:rec}
\lstinputlisting{src/receiver.vhdl}

\subsubsection{Simulation}
Figure \ref{fig:recsim} shows the simulation results. Listing \ref{lst:rectb} shows the testbench
used to run this simulation.
\begin{figure}[H]
  \centering
  \includegraphics[width=1.0\textwidth]{images/receivsim.eps}
  \caption{Receiver Simulation Waveform}
  \label{fig:recsim}
\end{figure}

\subsection{LCD Display Module}
The LCD available on the Spartan3E starter board is a 2-line by 16-character LCD. It is controlled through
a 4 bit interface, along with an enable, register select, and R/W line, named LCD\_E, LCD\_RS, and LCD\_RW.
LCD\_E is the Read/Write enable pusle. LCD\_RS selects the register, 0 being the instruction register during
write operations, and 1 being data for read or write operations. LCD\_RW will write on 0, and read on 1 into
the flash  memory.\cite{xilinxug}

\subsection{Resources}
\begin{tabular}{c | c | c | c}
  Occupied Slices Used & Occupied Slices Available & 4 Input LUTs Used & 4 Input LUTs available \\ \hline
  374 & 4656 & 681 & 9312 \\
\end{tabular}
\section{Discussion}
\subsection{Design Challenges}
One of the main challenges was getting a design that properly simulated to synthesize correctly. It seems there are
a few guidelines you have to follow in order for a design to be synthesizable and function correctly. Unfortunately,
have to keep synthesizing a design and testing it is a time consuming task. ModelSim provides an option to check
whether a design is synthesizable.\cite{modelsim} The ``-check\_synthesis'' flag is used to do this. It also happened that 
ModelSim's tests passed, and the design still did not function correctly on the board. It seems that warnings for
Xilin's XST can cause a design to malfunction. Its very important to check the output for latch warnings, which
seem to occur when an output is not assigned under all conditions.\cite{xilinx} Once these problems were taken care of, the
design functioned correctly.

An issue that came up during the testing phase was that of FERR. The initial design did not account for that fact that
the clock in the receiver is pulsing 16 times faster than the clock in the transmitter. Initially, the receiver would
only output a FERR for one of its clock cycles. This meant that when transferring 1bps, it would only last for $1/16^{th}$
of a second. This was overcome by having the receiver stay in an error state for 16 clock cycles, which was good enough
for transferring at 1bps. A better design would have been selecting a longer wait state when choosing the 10bps transfer rate.

\section{Suggestions}
There are many possible additions for this project. The system can be expanded in many ways. One important addition would be
to store the received data and status bits in a register. This register could be cleared once the user has read from it. This
would also give the ability to add additional status signals for things like notification of writing over an unread byte of
data.

Additionally, error correcting codes could be added to the design. There are two possible ways to approach this. First, it
could be a purely software approach in which the programmer is required to check for errors. A better approach would be to
implement in hardware. To do this, the number of data bits sent per transaction would need to be increased. A problem here
might be in interfacing with existing devices. Looking through the man pages for stty, there does not seem to be a way
of increasing the number of data bits per transaction past 8. We can either use multiple transactions, or, settle for
error detection using a parity bit, which is an option. The following is a portion of the stty man pages that would
be relevant in making design considerations if interfacing with the computer was important:
\begin{verbatim}
   Special settings:
       N      set the input and output speeds to N bauds

       * cols N
              tell the kernel that the terminal has N columns

       * columns N
              same as cols N

       ispeed N
              set the input speed to N

       * line N
              use line discipline N

       min N  with -icanon, set N characters minimum for a completed read

       ospeed N
              set the output speed to N

       * rows N
              tell the kernel that the terminal has N rows

       * size print the number of rows and columns according to the kernel

       speed  print the terminal speed

       time N with -icanon, set read timeout of N tenths of a second

   Control settings:
       [-]clocal
              disable modem control signals

       [-]cread
              allow input to be received

       * [-]crtscts
              enable RTS/CTS handshaking

       csN    set character size to N bits, N in [5..8]

       [-]cstopb
              use two stop bits per character (one with `-')

       [-]hup send a hangup signal when the last process closes the tty

       [-]hupcl
              same as [-]hup

       [-]parenb
              generate parity bit in output and expect parity bit in input

       [-]parodd
              set odd parity (even with `-')
\end{verbatim}
The man pages say that the size can be set anywhere for 5 to 8. Since 12 bits are needed for one byte of data
to support error correction, a good solution would be to create a protocol on top of RS232. It can specify
that 1 byte is sent in 2 packets, requiring two transactions. Making some sort of software library would be
a good idea as the communication protocol can be modulated.

Based on the man pages, its seems that implementing the handshaking mechanism would also be wise, especially if
interfacing with a user. Using a clock rate of 1 and 10 hz for transmission isnt really practical, so implementing
this would be nice.

\section{Alternative Serial Communication Systems}
This section discusses alternative serial communication systems.
\subsection{USB}
USB, like RS-232, is used to provide serial communication between a host and devices. USB provides speeds of 1.5Mbps,
12Mbps, and 140Mbps. Unlike RS-232, USB defines an application layer.
\subsection{SATA}
SATA, or serial advanced technology attachment, is a computer bus used for communicating between a host and mass
storage device. SATA is currently on its third generation and capable of operating at speeds up to 6Mbps. Whlie
conventional hard drives have no use for speeds of 6Mbps, sold state hard drives do. As they have no moving parts,
they have the capability of operating at this speed.
operates at 10Mbps.

\section{Conclusion}
This paper has described the implementation of a simple RS-232 based UART system. In it, a receiver and transmitter
have been described. 

\section{Additional Code}
\lstset{caption=Receiver Testbench,label=lst:rectb}
\lstinputlisting{src/testbench_receiver.vhdl}

\lstset{caption=System Testbench,label=lst:trantb}
\lstinputlisting{src/testbench_txrx.vhdl}

\lstset{caption=Clock Generator,label=lst:cg}
\lstinputlisting{src/clock_gen.vhdl}

\lstset{caption=Top Module,label=lst:top}
\lstinputlisting{src/top.vhdl}


%\section{Code}
%\lstset{ %
%language=VHDL,                % choose the language of the code
%basicstyle=\footnotesize,
%numbers=left,                   % where to put the line-numbers
%stepnumber=1,                   % the step between two line-numbers. If it's 1 each line will be numbered
%numbersep=10pt,                  % how far the line-numbers are from the code
%showspaces=false,               % show spaces adding particular underscores
%showstringspaces=false,         % underline spaces within strings
%showtabs=false,                 % show tabs within strings adding particular underscores
%frame=shadowbox,	                % adds a frame around the code
%tabsize=2,	                % sets default tabsize to 2 spaces
%captionpos=b,                   % sets the caption-position to bottom
%breaklines=true,                % sets automatic line breaking
%breakatwhitespace=false        % sets if automatic breaks should only happen at whitespace
%}

%\lstset{caption=PreLab Part 1,label=lst:p1}
%\lstinputlisting{src/d2a.m}

%\lstset{caption=Prelab Part 2,label=lst:p2}
%\lstinputlisting{src/counter.vhdl}

%%\lstset{caption=8 bit Fast Adder}
%%\lstinputlisting{./FastAdder8.vhd}

%%\lstset{caption=Testbench Module,label=lst:tb}
%%\lstinputlisting{./Testbench.vhd} 

\nocite{*}
\bibliographystyle{plain}
\bibliography{project}
\end{document}

